%-------------------------------------------------------------------------------
%	SECTION TITLE
%-------------------------------------------------------------------------------
\cvsection{Work Experience}
\begin{cventries}

	\cventry{Software Engineer (Pixel Graphics)} % Job title
	{Google} % Organization
	{London, United Kingdom} % Location
	{October 2025 -- Present} % Date(s)
	{
		\begin{cvitems} % Description(s) of tasks/responsibilities
			\item Learning about the Pixel GPUs and their supporting software.
			\item Working with senior engineers on the team to identify bugs and performance issues.
			\item Developing testing and analysis tools for Pixel GPUs.
			\item Working with teams across Pixel, Android, and Google to ensure that their software takes advantage of the Pixel GPU's capabilities.
			\item Collaborating with partners around improvements and new features.
		\end{cvitems}
	}

	\cventry{Software Engineer (C++, Objective-C, macOS and Windows desktop programming, CMake, Jenkins, C\#/Avalonia, Windows driver programming)} % Job title
	{PetaGene Ltd} % Organization
	{Cambridge, United Kingdom} % Location
	{August 2023 -- August 2025} % Date(s)
	{
		\begin{cvitems} % Description(s) of tasks/responsibilities
			\item Spearheaded build tools transition from GNU Autotools and shell script to vcpkg and CMake.
			Reduced build and CI run times by 10\(\times\).
			\item Introduced optimisations for thumbnails in Windows Explorer and macOS Finder by reverse-engineering system libraries with IDA Pro. Improved responsiveness for media and entertainment customers by reducing network traffic by 90\%. Used by a major film studio.
			\item Implemented local-file caching with macOS Finder extension.
			\item Applied static analysis and code quality tools like clang-tidy on the codebase, improving code quality, safety and maintainability.
			\item Represented the company at \href{https://hallerickson.ungerboeck.com/prod/app85.cshtml?aat=5663535078317a434974474478437845326c2b766b2b4c562b355033396d433556704e2b3065744c5161773d&ExhibitorID=7040}{SuperComputing 2023}, the largest HPC conference in the world in Denver, CO.\@ Attracted \textasciitilde100 leads, and several new customers.
		\end{cvitems}
	}

	%---------------------------------------------------------

	% \cventry{Undergraduate Teaching Assistant} % Job title
	% {National University of Singapore (NUS) School of Computing (SoC)} % Organization
	% {Singapore} % Location
	% {Aug 2020 -- Nov 2022} % Date(s)
	% {
	% 	\begin{cvitems} % Description(s) of tasks/responsibilities
	% 		\item Taught classes in computer graphics, real-time rendering, introductory programming, and computer architecture.
	% 		\item Set up auto-grading harness for computer graphics assignments to automate marking by comparing framebuffers and pixel errors.
	% 		\item Conducted weekly tutorials and recitations, prepared materials and videos for students, and marked assignments.
	% 	\end{cvitems}
	% }

	%---------------------------------------------------------

	\cventry{Embedded Software Engineering Intern (Sensors and IoT Division: C/C++, CMake, STM32)} % Job title
	{Government Technology Agency, Singapore (GovTech)} % Organization
	{Singapore} % Location
	{May 2022 -- August 2022} % Date(s)
	{
		\begin{cvitems} % Description(s) of tasks/responsibilities
			\item Implemented a C++ wrapper over Linux Serial Peripheral Interface (SPI) syscall interface. Reduced wheel-reinvention, and improved linkage for other projects using C++.
			\item Reused COVID-19 contact-tracing tokens using STM32 microcontrollers, by rewriting firmware in C++ to emulate a Trusted Platform Module (TPM) over I\textsuperscript{2}C for Raspberry Pi (rPi). Reduced costs by recycling and reusing existing hardware.
		\end{cvitems}
	}

	%---------------------------------------------------------

	%---------------------------------------------------------
\end{cventries}
