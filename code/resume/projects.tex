%-------------------------------------------------------------------------------
%	SECTION TITLE
%-------------------------------------------------------------------------------
\cvsection{Projects}
\vspace{1mm}

\cvsubsection{Personal projects and contributions}
\begin{cventries}
	\cventry{C++20 module for Vulkan-Hpp (C++, CMake)} % Description
	{vulkan.cppm \footnotesize{(\href{https://github.com/KhronosGroup/Vulkan-Hpp/pull/1582}{Merge request})}} % Project title
	{} % leave blank
	{} % leave blank
	{
		\begin{cvitems} % Project description
			\item Adapted a code generator to output a C++20 module interface file for the Vulkan-Hpp wrapper library.
			\item Improved type safety and performance by exporting C macros and function-like macros as \texttt{constexpr} variables and functions.
		\end{cvitems}
	}

	%---------------------------------------------------------

	% \cventry{C++ standard module arguments for Clang-cl driver} % Description
	% {LLVM \footnotesize{(\href{https://github.com/llvm/llvm-project/pull/98761}{PR \#98671}, \href{https://github.com/llvm/llvm-project/pull/121046}{PR \#121046})}} % Project title
	% {} % leave blank
	% {} % leave blank
	% {
	% 	\begin{cvitems} % Project description
	% 		\item Exposed \texttt{clang} C++ standard module compilation arguments to \texttt{clang-cl} driver.
	% 	\end{cvitems}
	% }
\end{cventries}

\cvsubsection{Coursework}
\begin{cventries}
	\cventry{Compiler for statically-typed, C-like \href{https://ilyasergey.net/CS4212/_static/oat-full.pdf}{Oat language} with Python-like list comprehension (OCaml, Menhir)} % Description
	{Oat Compiler} % Project title
	{} % leave blank
	{} % leave blank
	{
		\begin{cvitems} % Project description
			\item Front-end outputs a subset of LLVM IR;\@ back-end compiles IR to a subset of x86\_64 assembly.
			\item Includes compile-time type-checking and optimisations e.g.\ constant folding, dead-code elimination, and register allocation with graph colouring.
		\end{cvitems}
	}

	\cventry{} % Description
	{Also enjoyed working on, from scratch:} % Project title
	{} % leave blank
	{} % leave blank
	{
		\vspace{-4mm}
		\begin{cvitems}
		\item \textbf{\href{https://github.com/sharadhr/cs4223-cache-sim}{cache-sim}} (C++20, CMake), which implements a variety of cache-coherence protocols for a quad-core CPU;
		\item \textbf{Lexer and parser for a C-like toy language} (C++17), using recursive descent parsing and \texttt{std::regex} state machine.
	\end{cvitems}
	}
\end{cventries}

	%---------------------------------------------------------

	% \cventry{Quad-core cache-coherence simulator (C++20, CMake)} % Description
	% {cache-sim \footnotesize(\href{https://github.com/sharadhr/cs4223-cache-sim}{Repository})} % Project title
	% {} % leave blank
	% {} % leave blank
	% {
	% 	\begin{cvitems} % Project description
	% 		\item Implements MESI, MOESI, and Dragon cache-coherence protocols.
	% 		\item Correctly simulates cache-coherence behaviour of a real quad-core CPU, is configurable (cache size, associativity), and outputs statistics in \texttt{.csv} format.
	% 	\end{cvitems}
	% }

	%---------------------------------------------------------

	% \cventry{Lexer and parser for a C-like toy language (C++17)} % Description
	% {Static Program Analyser} % Project title
	% {} % leave blank
	% {} % leave blank
	% {
	% 	\begin{cvitems} % Project description
	% 		\item Lexer implemented with \texttt{std::regex} state machine; parser is recursive-descent.
	% 		\item Inserts information such as variable declarations, function calls, and control flow into a database about a given program written in the toy language.
	% 	\end{cvitems}
	% }

	%---------------------------------------------------------
